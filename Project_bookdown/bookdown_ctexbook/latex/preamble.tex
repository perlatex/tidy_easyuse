\usepackage{booktabs}
\usepackage{longtable}


\usepackage{array}
\usepackage{multirow}
\usepackage{wrapfig}
\usepackage{float}
\usepackage{colortbl}
\usepackage{pdflscape}
\usepackage{tabu}
\usepackage{threeparttable}
\usepackage{threeparttablex}
\usepackage[normalem]{ulem}
\usepackage{makecell}
\usepackage{xcolor}
\usepackage{indentfirst}\setlength{\parindent}{2em}
\usepackage{setspace}\doublespacing


%%%%%%%%%%%%%%%%%%%%%%%%%%%%%%%%%%%%%%%%%%%%%%%%%%%%%%%%%%%%%
%\setCJKmainfont[
%     BoldFont=方正黑体简体,
%   ItalicFont=方正精楷简体,
%  SlantedFont=方正仿宋简体,
%]{方正黑体简体}

%\setCJKsansfont[
%     BoldFont=方正黑体简体,
%   ItalicFont=方正精楷简体,
%  SlantedFont=方正仿宋简体,
%]{方正中等线简体}

%\setCJKmonofont[
%     BoldFont=方正黑体简体,
%   ItalicFont=方正精楷简体,
%  SlantedFont=方正仿宋简体,
%]{方正精楷简体}


\newCJKfontfamily{\song}{SimSun}
\newCJKfontfamily{\hei}{SimHei}
\newCJKfontfamily{\fzxbsj} {方正小标宋简体}
\newCJKfontfamily{\fzfangsong}{方正仿宋简体}
\newCJKfontfamily{\fzheiti}{方正黑体简体}
\newCJKfontfamily{\fzjk}{方正精楷简体}
\newCJKfontfamily{\fzcusong} {方正粗宋简体}
\newCJKfontfamily{\fzzhysong}{方正中雅宋_GBK}
\newCJKfontfamily{\fzydzhhei}{方正韵动中黑简体}
\newCJKfontfamily{\fzliukai} {方正苏新诗柳楷简体}
%%%%%%%%%%%%%%%%%%%%%%%%%%%%%%%%%%%%%%%%%%%%%%%%%%%%%%%%%%%%%

\usepackage{framed,color}
\definecolor{shadecolor}{RGB}{248,248,248}

\renewcommand{\textfraction}{0.05}
\renewcommand{\topfraction}{0.8}
\renewcommand{\bottomfraction}{0.8}
\renewcommand{\floatpagefraction}{0.75}

\let\oldhref\href
\renewcommand{\href}[2]{#2\footnote{\url{#1}}}

\makeatletter
\newenvironment{kframe}{%
\medskip{}
\setlength{\fboxsep}{.8em}
 \def\at@end@of@kframe{}%
 \ifinner\ifhmode%
  \def\at@end@of@kframe{\end{minipage}}%
  \begin{minipage}{\columnwidth}%
 \fi\fi%
 \def\FrameCommand##1{\hskip\@totalleftmargin \hskip-\fboxsep
 \colorbox{shadecolor}{##1}\hskip-\fboxsep
     % There is no \\@totalrightmargin, so:
     \hskip-\linewidth \hskip-\@totalleftmargin \hskip\columnwidth}%
 \MakeFramed {\advance\hsize-\width
   \@totalleftmargin\z@ \linewidth\hsize
   \@setminipage}}%
 {\par\unskip\endMakeFramed%
 \at@end@of@kframe}
\makeatother

\makeatletter
\@ifundefined{Shaded}{
}{\renewenvironment{Shaded}{\begin{kframe}}{\end{kframe}}}
\@ifpackageloaded{fancyvrb}{%
  % https://github.com/CTeX-org/ctex-kit/issues/331
  \RecustomVerbatimEnvironment{Highlighting}{Verbatim}{commandchars=\\\{\},formatcom=\xeCJKVerbAddon}%
}{}
\makeatother

\usepackage{makeidx}
\makeindex

\urlstyle{tt}

\usepackage{amsthm}
\makeatletter
\def\thm@space@setup{%
  \thm@preskip=8pt plus 2pt minus 4pt
  \thm@postskip=\thm@preskip
}
\makeatother


%%%%%%%%%%%%%%%%%%%%%%%%%%%%%%%%%%%%%%%%%%%%%%%%%%%%%%%%%%%%%%%%%%%%%%%%%%%%%%%%%%%
\newcommand\chaptertitleformat[1]{%% 设置条件语句,如果是无编号标题就再加一条横线
    \ifthechapter{#1}{#1\medskip\hrule\medskip}}

\makeatletter
\newcommand*\thechapteron{\global\let\ifthechapter\@firstoftwo}
\newcommand*\thechapteroff{\global\let\ifthechapter\@secondoftwo}
\thechapteroff
\makeatother



\ctexset{
    chapter = {
        format      = \fzzhysong\LARGE\bfseries,
        nameformat  = \thechapteron,                  %% 切换 \ifthechapter
        aftername   = \medskip\hrule\medskip,         %% 有编号标题的横线
        titleformat = \raggedleft\chaptertitleformat, %% 右对齐,无编号标题下再加横线
        aftertitle  = \thechapteroff
    },
    section = {
        format = \zihao{3}\raggedright\Large\bfseries,
    }
}

% caption settings
\usepackage[font=small,labelfont={bf}]{caption}
\captionsetup[table]{skip=3pt}
\captionsetup[figure]{skip=3pt}
%%%%%%%%%%%%%%%%%%%%%%%%%%%%%%%%%%%%%%%%%%%%%%%%%%%%%%%%%%%%%%%%%%%%%%%%%%%%%%%%%%%


\frontmatter
