\institute{A bookdown wrapper for ElegantBook}
\version{ElegantBook 3.11} % 当前使用的 elegantbook 宏包版本
\bioinfo{自定义}{信息}
\extrainfo{Victory won\rq t come to us unless we go to it. --- M. Moore}

% 字体设置
\setCJKmainfont[Path=../Fonts/,
                BoldFont={AdobeHeitiStd.otf},
                ItalicFont={AdobeKaitiStd.otf}]{AdobeSongStd.otf}
                
\setCJKsansfont[Path=../Fonts/,
                BoldFont={AdobeHeitiStd.otf},
                ItalicFont={AdobeHeitiStd.otf}]{AdobeHeitiStd.otf}
                
\setCJKmonofont[Path=../Fonts/,
                BoldFont={AdobeHeitiStd.otf},
                ItalicFont={AdobeHeitiStd.otf}]{AdobeFangsongStd.otf}

\setCJKfamilyfont{zhsong}[Path=../Fonts/]{AdobeSongStd.otf}
\setCJKfamilyfont{zhhei}[Path=../Fonts/]{AdobeHeitiStd.otf}
\setCJKfamilyfont{zhkai}[Path=../Fonts/]{AdobeKaitiStd.otf} 
\setCJKfamilyfont{zhfs}[Path=../Fonts/]{AdobeFangsongStd.otf}




\newcommand*{\songti}{\CJKfamily{zhsong}} 
\newcommand*{\heiti}{\CJKfamily{zhhei}} 
\newcommand*{\kaishu}{\CJKfamily{zhkai}} 
\newcommand*{\fangsong}{\CJKfamily{zhfs}}

% 下面如果不注释就准备好 Logo 和封面图片
% \logo{logo-blue.png} % 图片尺寸 1:1
% \cover{cover.jpg} % 图片尺寸 1280 × 1024

% Cancel common factors in Math 
\usepackage[makeroom]{cancel}

\usepackage[export]{adjustbox} %Needed for max width
%\patchcmd{<command>}{<code to replace>}{<code>}{<success>}{<failure>}
%The following codes add max dimension option to includegraphics
%Gin@ii is from graphicx package and looks for a second optional argument
\expandafter\patchcmd\csname Gin@ii\endcsname 
{\setkeys{Gin}{#1}}
{\setkeys{Gin}{max width=\textwidth, max height=.5\textwidth,keepaspectratio,#1}}
{}
{}

\definecolor{colortip}{RGB}{81,183,73}
\definecolor{colornote}{RGB}{251,188,5}
\definecolor{colorwarn}{RGB}{255,83,59}
\definecolor{colorinfo}{RGB}{204,204,204}

\tcbset{
  colbacktitle=white,
  enhanced,
  attach boxed title to top center={yshift=-2mm},
  colback=white, % 背景色
  coltext=black, % 文本色
  leftrule=1mm,
  rightrule=.25mm,
  bottomrule=.25mm,
  toprule=.25mm,
  boxsep=1pt, % 文字和边框的空隙
  arc=1pt % 圆角
}

\newtcolorbox{rmdtip}[1]{
  title=#1,
  coltitle=colortip,
  colframe=colortip, % 边框色
}

\newtcolorbox{rmdnote}[1]{
  title=#1,
  coltitle=colornote,
  colframe=colornote % 边框色
}

\newtcolorbox{rmdwarn}[1]{
  title=#1,
  coltitle=colorwarn,
  colframe=colorwarn % 边框色
}

\newtcolorbox{rmdinfo}{
  colframe=colorinfo % 边框色
}

\frontmatter
